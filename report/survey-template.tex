\documentclass[twocolumn]{article}

\title{Batch Messaging in Mixnets}

\author{name}

\date{}

%%packages
\usepackage[margin=1.5cm,nohead]{geometry}
\usepackage{graphicx}
\usepackage{hyperref}


\begin{document}

\maketitle

\begin{abstract}
The abstract contains:
\begin{itemize}
\item Description and motivation of survey topic (2-3 sentences)
\item Summary of results and how they can help future research (2-3 sentences) 
\end{itemize}
\end{abstract}


\section{Introduction}

The introduction contains:
\begin{itemize}
\item motivate why field of research is important (e.g., if you work on incentives for payment channel networks,  motivate why blockchain and payment channels are important) 
\item state concrete topic you are working on and motivate why the topic is important for the field 
\item state your key contributions/findings
\item shortly state how these findings can guide future research (~1 sentence)
\end{itemize}
While you don't need to include citations in the abstract,  from here on all the claims you make require evidence,  which can be citations or your own reasoning. Computer Science typically uses the IEEE referencing style, which can easily be achieved within LaTeX~\cite{brosius1995instructions,gray2018producing}. 

\section{Background}

While not required,  most papers will have a background section to explain concepts necessary to understand the remaining paper (e.g.,  routing in payment channel networks requires introducing some basics about payment channel networks).  You may assume that readers have a Bachelor degree in Computer Science and do not need to repeat elementary computer science concepts. 

\section{Main Section(s)}

Describe and assess the different papers in a meaningful order. For instance,  if there are k approaches,  have k sections.  You can also go from the most simple solution to the most sophisticated one.   The order depends on your papers and how they relate to each other, so there is no universal solution for deciding on the order. 

\section{Discussion}
In the discussion section,  you elaborate on the fundamental differences between the papers you introduced. 
Smaller aspects are discussed in the main sections.  In addition to the comparison,  you identify fundamental weaknesses and research gaps common to all approaches. 
A discussion section is not always necessary, sometimes it is easier to have the discussion during the main sections and then highlight the key results and limitations in the conclusion. 

\section{Conclusion}
The conclusion states the lessons learned and then explains how the knowledge gained in the paper can lead to future improvements. It suggest avenues for future work. These avenues should be larger unsolved questions regarding the topic as a whole,  not small improvements needed for  individual projects (these can go into the main sections). 


\bibliographystyle{plain}
\bibliography{survey}

\end{document}